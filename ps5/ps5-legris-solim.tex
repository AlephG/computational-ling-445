\documentclass[10pt]{article}
 
\usepackage[margin=1in]{geometry} 
\usepackage{amsmath,amsthm,amssymb, graphicx, multicol, array, parskip}
\usepackage[T1]{fontenc}
\usepackage{textcomp}
\usepackage{url}
% use straight quotes in texttt environment, make 0 different from O
\usepackage[zerostyle=b,straightquotes,scaled=.93]{newtxtt}

\usepackage{listings}
\lstset{
 basicstyle=\ttfamily,
 columns=fullflexible,
 upquote,
 keepspaces,
 literate={*}{{\char42}}1
 {-}{{\char45}}1
}
\usepackage[short labels]{enumitem}

\usepackage{soul}  % for strike through (\st{})
\usepackage[dvipsnames,usenames,table]{xcolor} % for colors
% \usepackage[htt]{hyphenat} % to break lines in texttt

\usepackage[framemethod=TikZ]{mdframed}
\usepackage{mdframed}
\global\mdfdefinestyle{simple}{linewidth=1pt,skipabove=.5em}
\newenvironment{AnswerBox}{\begin{mdframed}[style=simple]}{\end{mdframed}}
\newcommand\defeq{\mathrel{\overset{\makebox[0pt]{\mbox{\normalfont\tiny\sffamily def}}}{=}}}
\usepackage[colorlinks]{hyperref} % \href{http://link.com}{link text}
\hypersetup{linkcolor=NavyBlue,citecolor=NavyBlue,filecolor=NavyBlue,urlcolor=NavyBlue}
\usepackage{dsfont}
\newcommand{\required}[1]{{\color{blue}{#1}}}
\newcommand{\email}[1]{\href{mailto:#1}{\texttt{#1}}}
\newcommand{\PSnum}{5}

\author{ 
  \textbf{Name}: Solim LeGris     %Put your name here.
, \textbf{McGill ID}: 260807111   %Put your McGill ID here.
}

\begin{document}

\title{LING/COMP 445, LING 645\\Problem Set \PSnum}
\date{Due before 10:05 AM on Tuesday, November 23, 2021}
\maketitle
Please enter your name and McGill ID above.

This problem set consists only of questions involving mathematics or
English or or a combination of the two (no coding questions this time).
Please put your answers in an answer box like in the example below.

Once you have entered your answers, please compile your copy of this
\LaTeX{} into a PDF and submit 
\begin{enumerate}[(i),noitemsep]
\item
the compiled PDF renamed to
\texttt{ps\PSnum-lastname-firstname.pdf} and
\item
the raw \LaTeX{} file renamed to
\texttt{ps\PSnum-lastname-firstname.tex}
\end{enumerate}
to the Problem Set \PSnum{} folder under `Assignments' on MyCourses.


\hrulefill %--------------------------------

\paragraph{Example Problem:}
This is an example question using some fake math like this
$L=\sum_0^{\infty} \mathcal{G} \delta_x$.

\paragraph{Example Answer:} Put your answer in the box provided, like this:
\begin{AnswerBox}
Example answer is $L=\sum_0^{\infty} \mathcal{G} \delta_x$.
\end{AnswerBox}


\hrulefill%--------------------------------

\pagebreak%

\hrulefill %--------------------------------

\paragraph{Problem 1:}
 
In class we gave the following equation for the bigram probability of
a sequence of words $W^{(1)},\dots,W^{(k)}$:

\begin{equation}\label{eq:bigram}
\Pr(W^{(1)},\dots,W^{(k)})
\defeq\prod_i^k \Pr(W^{(i)} | W^{(i-1)}=w^{(i-1)})
\end{equation}

 Using this formula, give an expression for the bigram
probability of the sentence $abab$, where each character is treated as
a word. Try to simplify the formula as much as possible.

\paragraph{Important note:} Throughout this problem set, the vocabulary will be
$V\defeq\{a,b\}$. We will assume the length of the sentence is fixed at some $k$,
and \emph{we will not use the stop symbol}.  
That is, in a sentence of length $k$, for $1 \le i \le k$, the possible
values for the random variable $W^{(i)}$ are just $a$ and $b$, and we 
will refer to the beginning of the string as  $W^{(0)} = \rtimes$ always.
So, $\Pr(W^{(1)}=a \mid W^{(0)}=\rtimes)$ is
the probability that the string starts with $a$.

\paragraph{Answer 1:} Please put your answer in the box below.

\begin{AnswerBox}%% Do not delete %%%%%%%
\begin{eqnarray*}\label{bigram:abab}
    \Pr(W^{(1)}=a, W^{(3)}=b, W^{(3)}=a, W^{(4)}=b) & = & \Pr(W^{(1)}=a \mid W^{(0)}=\rtimes) \\
    & & \times \Pr(W^{(2)}=b \mid W^{(1)}=a) \\
    & & \times \Pr(W^{(3)}=a \mid W^{(2)}=b) \\
    & & \times \Pr(W^{(4)}=b \mid W^{(3)}=a) \\
    & = & \Pr(W^{(1)}=a) \\
    & & \times  \Pr(W^{(2)}=b \mid W^{(1)}=a) \\
    & & \times \Pr(W^{(3)}=a \mid W^{(2)}=b) \\
    & & \times \Pr(W^{(4)}=b \mid W^{(3)}=a)
\end{eqnarray*}
\end{AnswerBox}%% Do not delete %%%%%%%

\hrulefill %--------------------------------

\paragraph{Problem 2:}

There are two possible symbols/words in our language, $a$ and $b$. There are
three conditional distributions in the bigram model for this language, 
$\Pr(W^{(i)} | W^{(i-1)}=a)$, $\Pr(W^{(i)} | W^{(i-1)}=b)$, and
$\Pr(W^{(i)} | W^{(i-1)}=\rtimes)$. 
These conditional distributions are associated with the
parameter vectors $\vec{\theta}_{a}$, $\vec{\theta}_{b}$, and
$\vec{\theta}_{\rtimes}$, respectively (these parameter vectors were implicit in the
previous problem). For the current problem, we will assume that these
parameters are fixed. Use a second subscript notation to denote components of
these vectors, so $\theta_{ab}= \Pr(W^{(i)}=b\mid W^{(i-1)}=a)$.

 Suppose that we are given a sentence $W^{(1)},\dots,W^{(k)}$. We will
use the notation $n_{x \rightarrow y}$ to denote the number of times
that the symbol $y$ occurs immediately following the symbol $x$ in the
sentence. For example, $n_{a \rightarrow a}$ counts the number of
times that symbol $a$ occurs immediately following the symbol $a$.
Using Equation \ref{eq:bigram}, give an expression for the probability
of a length $k$ sentence in our language:
\begin{equation*}
\Pr(W^{(1)},\dots,W^{(k)} | \vec{\theta}_{a}, \vec{\theta}_{b}, \vec{\theta}_{\rtimes})
\end{equation*}

 The expression should make use of the $n_{x \rightarrow y}$ notation
defined above.

 Hint: the expression should be analogous to the formula that we found
for the likelihood of a corpus under a bag of words model.

\paragraph{Answer 2:} Please put your answer in the box below.

\begin{AnswerBox}%% Do not delete %%%%%%%
With $V=\{a,b\}$, 
\begin{equation}\label{eq:prob_se}
    \Pr(W^{(1)},\dots,W^{(k)} | \vec{\theta}_{a}, \vec{\theta}_{b}, \vec{\theta}_{\rtimes}) = \prod_{w\in V} \theta_{aw}^{n_{a \rightarrow w}} \theta_{bw}^{n_{b \rightarrow w}} \theta_{\rtimes w}^{n_{\rtimes \rightarrow w}}
\end{equation}
\end{AnswerBox}%% Do not delete %%%%%%%

\hrulefill %--------------------------------

\paragraph{Problem 3:}


Assume the parameter vectors in our bigram model have the following values:
\begin{align*}
&\vec{\theta}_{a} = (0.7,0.3)\\
&\vec{\theta}_{b} = (0.2,0.8)\\
&\vec{\theta}_{\rtimes} = (0.5,0.5)
\end{align*}

 The first vector indicates that if the current symbol $a$,
there is probability $0.7$ of transitioning to the symbol $a$, and
probability $0.3$ of transitioning to the symbol $b$. Using your
answer to the previous problem and these parameter values, calculate
the probability of the string $aabab$.

\paragraph{Answer 3:} Please put your answer in the box below.


\begin{AnswerBox}%% Do not delete %%%%%%%

\begin{align*}
    \Pr(W^{(1)} = a,W^{(2)} = a, W^{(3)} = b, W^{(4)} = a, W^{(5)} = b | \vec{\theta}_{a}, \vec{\theta}_{b}, \vec{\theta}_{\rtimes}) & = 0.5\times 0.7\times 0.3^2 \times 0.2 \\
    & = \frac{63}{10000}
\end{align*}
    
\end{AnswerBox}%% Do not delete %%%%%%%


\hrulefill %--------------------------------

\paragraph{Problem 4:}

In Problem 2, you found an expression for the bigram probability of a
sentence in our language, which contains the symbols $a$ and $b$. In
that problem, we assumed that there were fixed parameter vectors
$\vec{\theta}$ associated with each conditional distribution. In this
problem, we will consider the setting in which we have uncertainty
about the value of these parameters.

 As we did in class, we will use the Dirichlet distribution to 
define a prior over parameters.  Assume each parameter vector is drawn
independently given $\vec{\alpha}$:
\begin{align}
\vec{\theta}_{\mathbf{c}} \mid \vec{\alpha} 
    &\sim\mathrm{Dirichlet}(\vec{\alpha}) \\
w^{(i)} \mid  w^{(i-1)} 
    &\sim\mathrm{categorical}(\vec{\theta}_{w^{(i-1)}}) \\
w^{(1)} 
    &\sim \mathrm{categorical}(\vec{\theta}_{\rtimes})\
\end{align}

 Suppose that we have a fixed-length sentence
$S=W^{(1)},\dots,W^{(k)}$. Give an expression for the joint
probability
$\Pr(S, \vec{\theta}_{a}, \vec{\theta}_{b}, \vec{\theta}_{\rtimes} |
\vec{\alpha})$
using the definitions of Dirichlet distributions and likelihoods we
defined in class.

\paragraph{Answer 4:} Please put your answer in the box below.


\begin{AnswerBox}%% Do not delete %%%%%%%
The joint probability can be expressed as follows, \\
\begin{align*}
    \Pr(S, \vec{\theta}_{a}, \vec{\theta}_{b}, \vec{\theta}_{\rtimes} | \vec{\alpha}) & = \Pr(S|\vec{\theta}_{a}, \vec{\theta}_{b}, \vec{\theta}_{\rtimes})\Pr(\vec{\theta}_{a}, \vec{\theta}_{b}, \vec{\theta}_{\rtimes}|\vec{\alpha})
\end{align*}
We know $\Pr(S|\vec{\theta}_{a}, \vec{\theta}_{b}, \vec{\theta}_{\rtimes})$ from Equation \ref{eq:prob_se} derived in Problem 2. Now, knowing that each parameter vector is drawn independently given $\vec{\alpha}$, we can express the prior using the following:

\begin{align*}
    \Pr(\vec{\theta}_{a}|\vec{\alpha}) & =  \frac{\Gamma (\sum_{w\in V} \alpha_{w})}{\prod_{w\in V} \Gamma (\alpha_{w})}\prod_{w\in V}\theta_{aw}^{\alpha_{w} -1}\\
    \Pr(\vec{\theta}_{b}|\vec{\alpha}) & =  \frac{\Gamma (\sum_{w\in V} \alpha_{w})}{\prod_{w\in V} \Gamma (\alpha_{w})}\prod_{w\in V}\theta_{bw}^{\alpha_{w} -1}\\
    \Pr(\vec{\theta}_{\rtimes}|\vec{\alpha}) & =  \frac{\Gamma (\sum_{w\in V} \alpha_{w})}{\prod_{w\in V} \Gamma (\alpha_{w})}\prod_{w\in V}\theta_{\rtimes w}^{\alpha_{w} -1}
\end{align*}
Using these equations, we can now define our joint probability equation,

\begin{align}\label{eq:joint}
    \Pr(S, \vec{\theta}_{a}, \vec{\theta}_{b}, \vec{\theta}_{\rtimes} | \vec{\alpha}) & = 
     \left (\frac{\Gamma (\sum_{w\in V} \alpha_{w})}{\prod_{w\in V} \Gamma (\alpha_{w})}\right )^3 \prod_{w\in V} \theta_{aw}^{n_{a \rightarrow w}+ \alpha_{w} -1} \theta_{bw}^{n_{b \rightarrow w}+\alpha_{w} -1} \theta_{\rtimes w}^{n_{\rtimes \rightarrow w}+\alpha_{w} -1}
\end{align}



    
\end{AnswerBox}%% Do not delete %%%%%%%


\hrulefill %--------------------------------

\paragraph{Problem 5:}

In the previous problem, you found a formula for the joint probability
of a sentence and a set of bigram model parameters. Using this, give a
formula for the marginal probability of the sentence
$\Pr(S|\vec{\alpha})$.

 Hint: The formula should be analogous to the formula derived
in class for marginal probability of a corpus under a bag of words
model. Whereas before there was only a single parameter vector
$\vec{\theta}$, now there are three parameter vectors that need to be
marginalized away. Otherwise the calculation will be similar.

\paragraph{Answer 5:} Please put your answer in the box below.


\begin{AnswerBox}%% Do not delete %%%%%%%

\begin{align*}
    \Pr(S|\vec{\alpha}) & =  \left (\frac{\Gamma (\sum_{w\in V} \alpha_{w})}{\prod_{w\in V} \Gamma (\alpha_{w})}\right )^3 
    \int_{\Theta_{a}} \int_{\Theta_{b}} 
    \int_{\Theta_{\rtimes}} 
    \prod_{w\in V} \theta_{aw}^{n_{a \rightarrow w}+ \alpha_{w} -1} \theta_{bw}^{n_{b \rightarrow w}+\alpha_{w} -1} \theta_{\rtimes w}^{n_{\rtimes \rightarrow w}+\alpha_{w} -1}d\theta_a d\theta_b d\theta_{\rtimes}\\
    & = \left (\frac{\Gamma (\sum_{w\in V} \alpha_{w})}{\prod_{w\in V} \Gamma (\alpha_{w})}\right )^3 
    \left (\frac{\prod_{w\in V} \Gamma (n_{a \rightarrow w}+ \alpha_{w})}{\Gamma (\sum_{w\in V} n_{a \rightarrow w}+ \alpha_{w})}\right)
    \left (\frac{\prod_{w\in V} \Gamma (n_{b \rightarrow w}+ \alpha_{w})}{\Gamma (\sum_{w\in V} n_{b \rightarrow w}+ \alpha_{w})}\right)
    \left (\frac{\prod_{w\in V} \Gamma(n_{\rtimes \rightarrow w}+ \alpha_{w})}{\Gamma (\sum_{w\in V} n_{\rtimes \rightarrow w}+ \alpha_{w})}\right)\\ 
    %% Maybe this can be expressed as a product over (a,b,start)
\end{align*}

To simplify, we express the former using a product over the set $s \in \{a,b,\rtimes \}$,
\begin{equation}\label{eq:marginal_data}
      \Pr(S|\vec{\alpha})  = \left (\frac{\Gamma (\sum_{w\in V} \alpha_{w})}{\prod_{w\in V} \Gamma (\alpha_{w})}\right )^3 
    \prod_{s\in V\cup \{\rtimes\}} \left (\frac{\prod_{w\in V} \Gamma (n_{s \rightarrow w}+ \alpha_{w})}{\Gamma (\sum_{w\in V} n_{s \rightarrow w}+ \alpha_{w})}\right)
\end{equation}

    
\end{AnswerBox}%% Do not delete %%%%%%%


\hrulefill %--------------------------------

\paragraph{Problem 6:}


Let us assume that the parameters of the Dirichlet distribution are
$\vec{\alpha} = (1,1)$. Using your solution to the previous problem,
write an expression for $\Pr(S=aabab\mid \vec \alpha = (1,1))$,
the marginal probability of the string $aabab$.
The expression should should contain the
\href{https://en.wikipedia.org/wiki/Gamma_function}{gamma function}
$\Gamma(\cdot)$. Using the properties of the gamma function discussed
in class (i.e., it's relationship to the factorial) or an online
calculator, compute a numerical value for this expression.



\paragraph{Answer 6:} Please put your answer in the box below.


\begin{AnswerBox}%% Do not delete %%%%%%%
\begin{align*}
      \Pr(S=aabab|\vec{\alpha}=\langle 1, 1 \rangle)  & = \left (\frac{\Gamma (2)}{\Gamma (1) \Gamma (1)}\right )^3 \left (\frac{\Gamma (2)\Gamma (3)}{\Gamma (5)}\right)\left(\frac{\Gamma (2)\Gamma (1)}{\Gamma (3)}\right)\left(\frac{\Gamma (2)\Gamma (1)}{\Gamma (3)}\right)\\
      & = \left(\frac{1}{1}\right)^3 \left(\frac{2}{24}\right)\left( \frac{1}{2}\right)\left( \frac{1}{2}\right)\\
      & = \frac{1}{48}
\end{align*}
\end{AnswerBox}%% Do not delete %%%%%%%

\hrulefill %--------------------------------

\paragraph{Problem 7:}

Suppose that we have observed a sentence
$S=W^{(1)},\dots,W^{(k)}$. Find an expression for the posterior
distribution over the model parameters,
$\Pr(\vec{\theta_a}, \vec{\theta_b}, \vec{\theta}_{\rtimes} \mid S,\vec{\alpha})$.

 Hint: Use the joint probability that you computed in Problem
4 and Bayes' rule. The solution should be analogous to the posterior
probability for the bag of words model.

\paragraph{Answer 7:}  Please put your answer in the box below.


\begin{AnswerBox}%% Do not delete %%%%%%%

\begin{align*}
    \Pr(\vec{\theta_a}, \vec{\theta_b}, \vec{\theta}_{\rtimes} \mid S,\vec{\alpha}) & = \frac{\left (\frac{\Gamma (\sum_{w\in V} \alpha_{w})}{\prod_{w\in V} \Gamma (\alpha_{w})}\right )^3 \prod_{w\in V} \theta_{aw}^{n_{a \rightarrow w}+ \alpha_{w} -1} \theta_{bw}^{n_{b \rightarrow w}+\alpha_{w} -1} \theta_{\rtimes w}^{n_{\rtimes \rightarrow w}+\alpha_{w} -1}}{\left (\frac{\Gamma (\sum_{w\in V} \alpha_{w})}{\prod_{w\in V} \Gamma (\alpha_{w})}\right )^3 
    \left (\frac{\prod_{w\in V} \Gamma(n_{a \rightarrow w}+ \alpha_{w})}{\Gamma (\sum_{w\in V} n_{a \rightarrow w}+ \alpha_{w})}\right)
    \left (\frac{\prod_{w\in V} \Gamma (n_{b \rightarrow w}+ \alpha_{w})}{\Gamma (\sum_{w\in V} n_{b \rightarrow w}+ \alpha_{w})}\right)
    \left (\frac{\prod_{w\in V} \Gamma (n_{\rtimes \rightarrow w}+ \alpha_{w})}{\Gamma (\sum_{w\in V} n_{\rtimes \rightarrow w}+ \alpha_{w})}\right)} \\
    \Pr(\vec{\theta_a}, \vec{\theta_b}, \vec{\theta}_{\rtimes} \mid S,\vec{\alpha}) & = \frac{\prod_{w\in V} \theta_{aw}^{n_{a \rightarrow w}+ \alpha_{w} -1} \theta_{bw}^{n_{b \rightarrow w}+\alpha_{w} -1} \theta_{\rtimes w}^{n_{\rtimes \rightarrow w}+\alpha_{w} -1}}{ 
    \left (\frac{\prod_{w\in V} \Gamma (n_{a \rightarrow w}+ \alpha_{w})}{\Gamma (\sum_{w\in V} n_{a \rightarrow w}+ \alpha_{w})}\right)
    \left (\frac{\prod_{w\in V} \Gamma (n_{b \rightarrow w}+ \alpha_{w})}{\Gamma (\sum_{w\in V} n_{b \rightarrow w}+ \alpha_{w})}\right)
    \left (\frac{\prod_{w\in V} \Gamma (n_{\rtimes \rightarrow w}+ \alpha_{w})}{\Gamma (\sum_{w\in V} n_{\rtimes \rightarrow w}+ \alpha_{w})}\right)}
\end{align*}

\begin{align}
    \Pr(\vec{\theta_a}, \vec{\theta_b}, \vec{\theta}_{\rtimes} \mid S,\vec{\alpha}) & = \frac{\prod_{w\in V} \theta_{aw}^{n_{a \rightarrow w}+ \alpha_{w} -1} \theta_{bw}^{n_{b \rightarrow w}+\alpha_{w} -1} \theta_{\rtimes w}^{n_{\rtimes \rightarrow w}+\alpha_{w} -1}}{\prod_{s\in V\cup \{\rtimes\}} \left (\frac{\prod_{w\in V} \Gamma (n_{s \rightarrow w}+ \alpha_{w})}{\Gamma (\sum_{w\in V} n_{s \rightarrow w}+ \alpha_{w})}\right)}
\end{align}
    
\end{AnswerBox}%% Do not delete %%%%%%%


\hrulefill %--------------------------------

\paragraph{Problem 8:}

Using your formula for the posterior distribution and setting
$\vec{\alpha} = (1,1)$, given an expression for the posterior
distribution over parameters given the sentence $aabab$. There should
be an easy way to interpret the posterior distribution, and how it was
derived from the observed sentence. What is this interpretation?

\paragraph{Answer 8:} Please put your answer in the box below.


\begin{AnswerBox}%% Do not delete %%%%%%%

\begin{align*}
    \Pr(\vec{\theta_a}, \vec{\theta_b}, \vec{\theta}_{\rtimes} \mid S =aabab,\vec{\alpha}=\langle 1, 1 \rangle) & = \frac{\prod_{w\in V} \theta_{aw}^{n_{a \rightarrow w}} \theta_{bw}^{n_{b \rightarrow w}} \theta_{\rtimes w}^{n_{\rtimes \rightarrow w}}}{\prod_{s\in V\cup \{\rtimes\}} \left (\frac{\prod_{w\in V} \Gamma (n_{s \rightarrow w}+ 1)}{\Gamma (\sum_{w\in V} n_{s \rightarrow w}+ 1)}\right)}\\
    & =   \frac{\theta_{aa} \theta_{ba} \theta_{\rtimes a} \theta_{ab}^{2} \theta_{bb}^{0} \theta_{\rtimes b}^{0} }{ 
    (\frac{\Gamma (2)\Gamma (3)}{\Gamma (5)})
    (\frac{\Gamma(2)\Gamma (1)}{\Gamma (3)})
    (\frac{\Gamma(2)\Gamma (1)}{\Gamma (3)})}\\
    &= \frac{\theta_{aa} \theta_{ba} \theta_{\rtimes a} \theta_{ab}^{2}}{ 
    (\frac{2}{24})
    (\frac{1}{2})
    (\frac{1}{2})}\\
    &= \frac{\theta_{aa} \theta_{ba} \theta_{\rtimes a} \theta_{ab}^{2}}{ 
    (\frac{1}{48})}\\
    &= 48 \times \theta_{aa} \theta_{ba} \theta_{\rtimes a} \theta_{ab}^{2}
\end{align*}
    The posterior distribution is obtained using Bayes' rule and it was derived using the evidence (Equation \ref{eq:marginal_data}) and the joint distribution (Equation \ref{eq:joint}). Plugging in different values of model parameters in this equation will tell us how well these values fit this data.
\end{AnswerBox}%% Do not delete %%%%%%%


\hrulefill %--------------------------------

\paragraph{Problem 9:}

Consider the language $L=\{a^* b a^*\}$, that is, the language
consisting of some number of a's, followed by a single b, followed by
some number of a's. Show that this language is not strictly
$2$-local.

 Hint: use $n$-Local Suffix Substitution Closure ($n$-SSC).

\paragraph{Answer 9:} Please put your answer in the box below.


\begin{AnswerBox}%% Do not delete %%%%%%%
\begin{proof}
    Let $s_1 = \rtimes aaabaa \ltimes$ with $u_1 = \rtimes aaab$, $x=a$ and $u_2=a\ltimes$. Let $s_2=\rtimes aaba$ with $v_1=\rtimes$, $x=a$ and $v_2=aba\ltimes$. Our new string is $\rtimes aaababa \ltimes \notin L \Rightarrow \neg 2\mathrm{-SSC}(L) \Rightarrow L \notin \mathrm{SL}_2$ 
\end{proof}
    
\end{AnswerBox}%% Do not delete %%%%%%%


\hrulefill %--------------------------------

\paragraph{Problem 10:}

Consider the language
$L= \{a^n b^m c^n d^m\ | n, m \in \mathbb{N} \}$, that is, the
language consisting of $n$ a's followed by $m$ b's followed by $n$ c's
followed by $m$ d's where $n$ and $m$ are
natural numbers. Show that this language is not strictly $2$-local.

 Hint: use the same property as in the problem above.

\paragraph{Answer 10:} Please put your answer in the box below.


\begin{AnswerBox}%% Do not delete %%%%%%%
\begin{proof}
    Let $s_1 = \rtimes abcd \ltimes$ with $u_1 = \rtimes ab$, $x=c$ and $u_2=d\ltimes$. Let $s_2=\rtimes aabccd\ltimes$ with $v_1=\rtimes aab$, $x=c$ and $v_2=cd\ltimes$. Our new string is $\rtimes aabd \ltimes \notin L \Rightarrow \neg 2\mathrm{-SSC}(L) \Rightarrow L \notin \mathrm{SL}_2$ 
\end{proof}

\end{AnswerBox}%% Do not delete %%%%%%%


\hrulefill %--------------------------------

\paragraph{Problem 11:}

Show that the language $L= \{a^n b^m c^n d^m\ | n, m \in \mathbb{N} \}$ is not
strictly $k$-local, for any value of $k$.

\paragraph{Answer 11:} Please put your answer in the box below.


\begin{AnswerBox}%% Do not delete %%%%%%%
\begin{proof}
    Let $k \in \mathbb{N}$ and then choose $n=k$. Take a string in $L$ such that $s_1 = a^nb^{n+1}c^nd^{n+1}$ with $u_1=a^n$, $x=b^{n-1}$ and $u_2=b^2c^nd^{n+1}$ and another string $s_2 = a^{n+1}b^{n}c^{n+1}d^{n}$ with $u_1=a^{n+1}$, $x=b^{n-1}$ and $u_2=bc^{n+1}d^{n}$. The resulting string is $s_3=a^nb^{n-1}bc^{n+1}d^n = a^nb^{n}c^{n+1}d^n \notin L \Rightarrow \neg k\mathrm{-SSC}(L)  \Rightarrow L \notin \mathrm{SL}_k$ for any $k\in \mathbb{N}$ 
\end{proof}

\end{AnswerBox}%% Do not delete %%%%%%%


\hrulefill %--------------------------------

\paragraph{Problem 12:}

In class we proved that
$k\mathrm{-SSC}(L) \implies L \in \mathrm{SL}_k$. In other words, if a
language satisfies $k$-Local Suffix Substitution Closure, then it is
$k$-strictly local.

 Use this theorem to prove that $k$-strictly local languages
are closed under intersection. More precisely, prove that if
$L_1 \in \mathrm{SL}_k$ and $L_2 \in \mathrm{SL}_k$, then
$L_1 \cap L_2 \in \mathrm{SL}_k$.

\paragraph{Answer 12:}  Please put your answer in the box below.


\begin{AnswerBox}%% Do not delete %%%%%%%

\begin{proof}
    Suppose $L_1 \in \mathrm{SL}_k$ and $L_2 \in \mathrm{SL}_k$ and take $s_1 \in L_1\cap L_2$ and $s_2\in L_1\cap L_2$ where $s_1=u_1xu_2$, $s_2=v_1xv_2$ and $|x|=k-1$. We know from proving it in class that $L_1$ and $L_2$ are closed under $k$-local suffix substitution. Now, consider the string $s_3=u_1xv_2$. Since $s_1\in L_1$, $s_2\in L_1$ and $k\mathrm{-SSC}(L_1)$, then $s_3\in L_1$ but $s_3\in L_2$ as well since $s_1\in L_2$, $s_2\in L_2$ and $k\mathrm{-SSC}(L_2)$.\\
    $\therefore s_3\in L_1\cap L_2 \Rightarrow k\mathrm{-SSC}(L_1\cap L_2) \Rightarrow L_1\cap L_2\in \mathrm{SL}_k$ 
\end{proof}
    
\end{AnswerBox}%% Do not delete %%%%%%%

\end{document}
